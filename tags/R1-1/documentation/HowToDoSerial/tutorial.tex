\documentclass[twoside]{article}

\usepackage{epsfig}
\usepackage{times}
\usepackage{html}
\usepackage{alltt}
\usepackage{verbatimfiles}

\textwidth=6.5in
\topmargin=0pt
\headheight=0pt
\textheight=8.6truein
\oddsidemargin=0in
\evensidemargin=0in
\footskip=40pt

\parindent=0pt
\parskip=0.5ex

\setcounter{tocdepth}{2}

\title{How to create EPICS device support for a simple serial device}
\author{W. Eric Norum \\ $<$\htmladdnormallink{norume@aps.anl.gov}{mailto:norume@aps.anl.gov}$>$}

\begin{document}
\begin{latexonly}
\pagenumbering{roman}
\maketitle
\tableofcontents
\cleardoublepage
\pagenumbering{arabic}
\end{latexonly}

\section{Introduction}
This tutorial provides step-by-step instructions on how to create
EPICS support for a simple serial device.
The steps are presented in a way that should make it possible to apply
them in
cookbook fashion to create support for other serial devices.
For comprehensive description of all the details of the I/O system
used here, refer to the
\htmladdnormallink{asynDriver}{../../asynDriver.html}
and
\htmladdnormallink{devGpib}{../../devGpib.html}
documentation.

This document isn't for the absolute newcomer though.  You must have
EPICS installed on a system somewhere and know how
to build and run the example application.
In particular you must have the following 
installed:
\begin{itemize}
\item EPICS R3.14.5 or higher
\item An up-to-date version of {\tt modules/asyn}
\end{itemize}
Serial devices can now be treated in much the
same way as GPIB (IEEE-488) devices.
Current versions of the EPICS asynDriver devGpib module contain
a special
'GPIB' driver capable of communicating with devices connected to serial
ports on the IOC or with devices connected through Ethernet/Serial converter
boxes.

I based this tutorial on the device support I wrote for a
CVI Laser Corporation AB300 filter wheel.
You're almost certainly interested in controlling some other device
so you won't be able to use the information directly.
I chose the AB300 as the basis for this tutorial since
the AB300 has a very limited command set, which keeps this document small,
and yet has commands which raise many
of the issues that you'll have to consider
when writing support for other devices.

\begin{htmlonly}
If you'd like to print this tutorial you can download a
\htmladdnormallink{PDF version}{tutorial.pdf}.
\end{htmlonly}

\section{Determine the required I/O operations}
The first order of business is to determine the set of operations the
device will have to perform.
A look at the AB300 documentation reveals that there are four
commands that must be supported.
Each command will
be associated with an EPICS process variable (PV) whose type must be
appropriate to the data transferred by the command.
The AB300 commands and
process variable record types I choose to associate with them are shown
in table~\ref{commandList}.
\begin{table}[bht]
\caption{AB300 filter wheel commands\label{commandList}}
\begin{center}
\begin{tabular}{|l|l|}
\hline
\multicolumn{2}{|c|}{CVI Laser Corporation AB300 filter wheel} \\
\hline
\multicolumn{1}{|c|}{Command}  &\multicolumn{1}{c|}{EPICS record type} \\
\hline
Reset                &longout \\
Go to new position   &longout \\
Query position       &longin \\
Query status         &longin \\
\hline
\end{tabular}
\end{center}
\end{table}
There are lots of other ways that the AB300 could be handled.  It might
be useful, for example,  to treat the filter position as
multi-bit binary records instead.

\section{Create a new application}
Now that the device operations and EPICS process variable types have
been chosen it's time to create a new EPICS application to provide
a place to perform subsequent software development.
The easiest way to do this is with the makeBaseApp.pl script supplied
with EPICS.

Here are the commands I ran.   You'll have to change the
{\tt /home/EPICS/R3.14.5} to the path to where your EPICS is installed.
If you're
not running on Linux you'll also have to change all the {\tt linux-x86} to
reflect the architecture you're using ({\tt solaris-sparc}, {\tt darwin-ppc}, etc.).
I built the application as a 'soft' IOC running on the host machine, but
the serial 'GPIB' driver also works on RTEMS and vxWorks.

\begin{alltt}
norume> {\bf mkdir AB300}
norume> {\bf cd AB300}
norume> {\bf /home/EPICS/R3.14.5/base/bin/linux-x86/makeBaseApp.pl -t ioc AB300}
norume> {\bf /home/EPICS/R3.14.5/base/bin/linux-x86/makeBaseApp.pl -i -t ioc AB300}
The following target architectures are available in base:
    RTEMS-pc386
    linux-x86
    solaris-sparc
    win32-x86-cygwin
    vxWorks-ppc603
What architecture do you want to use? {\bf linux-x86}
\end{alltt}

\section{Make some changes to the files in configure/}
Edit the {\tt configure/RELEASE} file which makeBaseApp.pl created and add an
entry describing the path to where you installed the EPICS ASYN module:
\begin{alltt}
ASYN=/home/EPICS/R3.14.5/modules/soft/asyn
\end{alltt}

Edit the {\tt configure/CONFIG} file which makeBaseApp.pl created and specify
the IOC architectures on which the application is to run.
I wanted the application to run as a soft IOC, so I uncommented the
\verb@CROSS_COMPILER_TARGET_ARCHS@ definition and set the definition
to be empty:
\begin{verbatim}
CROSS_COMPILER_TARGET_ARCHS =
\end{verbatim}

\section{Create the device support file}
The contents of the device support
file provide all the details of the communication between the device
and EPICS.
The easiest way to create a device support file is to copy
the skeleton device support file from the gpibCore module source directory to
your application source directory:
\begin{alltt}
norume> {\bf cd AB300App/src}
norume> {\bf cp /usr/EPICS/R3.14.5/modules/soft/asyn/devGpib/devSkeletonGpib.c devAB300.c}
\end{alltt}
Of course, device support for a device similar
to the one you're working with  provides an even easier starting point.

The remainder this section describes the changes that I made to the skeleton
file in order to support the AB300 filter wheel.
You'll have to modify the steps as appropriate for your device.

\subsection{Declare the DSET tables provided by the device support}
Since the AB300
provides only longin and longout records most of the \verb@DSET_@{\it xxx}
define statements can
be removed.  Because of the way that the device initialization is performed
you must define an analog-in DSET even if the device provides no analog-in
records (as is the case for the AB300).
\begin{verbatim}
#define DSET_AI    devAB300_ai
#define DSET_LI    devAB300_li
#define DSET_LO    devAB300_lo
\end{verbatim}

\subsection{Select timeout values}
The default value of \verb@TIMEWINDOW@  (2 seconds) is
reasonable for the AB300, but I increased the value of \verb@TIMEOUT@ to
5~seconds since the filter wheel can be slow in responding.
\begin{verbatim}
#define TIMEOUT     5.0
#define TIMEWINDOW  2.0
\end{verbatim}


\subsection{Clean up some unused values}
The skeleton file provides a number of
character string arrays.  None are needed for the AB300 so I just removed them.
Not much space would be wasted by just leaving them in place however.

\subsection{Declare the command array}
This is the hardest part of the job.  Here's where you have to figure
how to produce the command strings required to control the device
and how to convert the device responses into EPICS process variable
values.

Each command array entry describes the details of a single
I/O operation type.
The application database uses the index of the entry in the command array to
provide the link between the process variable and the I/O operation to
read or write that value.

The command array entries I created for the AB300 are shown below.  The
elements of each entry are described using the names from
the
\htmladdnormallink{GPIB documentation}{http://www.aps.anl.gov/modules/soft/asyn/devGgpib.html}.
\subsubsection{Command array index 0 -- Device Reset}
\begin{verbatim}
   {&DSET_LO, GPIBWRITE, IB_Q_HIGH, NULL, "\377\377\033", 10, 10,
       NULL, 0, 0, NULL, NULL, "\033"},
\end{verbatim}
\begin{description}
\item[dset] This command is associated with an longout record.
\item[type] A WRITE operation is to be performed.
\item[pri] This operation should be placed on the high-priority queue
of I/O requests.
\item[cmd] Because this is a GPIBWRITE operation this element is unused.
\item[format] The format string to generate the command to be sent to the device.  The first two bytes are
the RESET command, the third byte is the ECHO command.
The AB300 sends no response to a reset command
so I send the 'ECHO' to verify that the
device is responding.  The AB300 resets itself fast enough that it can see an
echo command immediately following the reset command.

Note that the process variable value is not used (there's no printf \verb@%@
format character in the command string).  The AB300 is reset whenever the
EPICS record is processed.
\item[rspLen] The size of the readback buffer.  Although only one readback byte is
expected I allow for a few extra bytes just in case.
\item[msgLen] The size of the buffer into which the command string is placed.  I allowed
a little extra space in case a longer command is used some day.
\item[convert] No special conversion function is needed.
\item[P1,P2,P3] There's no special conversion function so no arguments are needed.
\item[pdevGpibNames] There's no name table.
\item[eos] The end-of-string value used to mark the end of the readback operation.
\end{description}




\subsubsection{Command array index 1 -- Go to new filter position}
\begin{verbatim}
    {&DSET_LO, GPIBWRITE, IB_Q_LOW, NULL, "\017%c", 10, 10,
        NULL, 0, 0, NULL, NULL, "\030"},
\end{verbatim}
\begin{description}
\item[dset] This command is associated with an longout record.
\item[type] A WRITE operation is to be performed.
\item[pri] This operation should be placed on the high-priority queue
of I/O requests.
\item[cmd] Because this is a GPIBWRITE operation this element is unused.
\item[format] The format string to generate the command to be sent to the device.
The filter
position (1-6) can be converted to the
required command byte with the printf \verb@%c@ format.
\item[rspLen] The size of the readback buffer.  Although only two readback bytes are
expected I allow for a few extra bytes just in case.
\item[msgLen] The size of the buffer into which the command string is placed.  I allowed
a little extra space in case a longer command is used some day.
\item[convert] No special conversion function is needed.
\item[P1,P2,P3] There's no special conversion function so no arguments are needed.
\item[pdevGpibNames] There's no name table.
\item[eos] The end-of-string value used to mark the end of the readback operation.
\end{description}



\subsubsection{Command array index 2 -- Query filter position}
\begin{verbatim}
    {&DSET_LI, GPIBREAD, IB_Q_LOW, "\035", NULL, 0, 10,
        convertPositionReply, 0, 0, NULL, NULL, "\030"},
\end{verbatim}
\begin{description}
\item[dset] This command is associated with an longin record.
\item[type] A READ operation is to be performed.
\item[pri] This operation should be placed on the high-priority queue
of I/O requests.
\item[cmd] The command string to be sent to the device.
The AB300 responds to this command by sending back three bytes: the current
position, the controller status, and a terminating \verb@'\030'@.
\item[format] Because this is a GPIBREAD operation this element is unused.
\item[rspLen] There is no command echo to be read.
\item[msgLen] The size of the buffer into which the reply string is placed. 
Although only three reply bytes are
expected I allow for a few extra bytes just in case.
\item[convert] There's no sscanf format that can convert the reply from the
AB300 so a special conversion function must be provided.
\item[P1,P2,P3] The special conversion function requires no arguments.
\item[pdevGpibNames] There's no name table.
\item[eos] The end-of-string value used to mark the end of the read operation.
\end{description}



\subsubsection{Command array index 3 -- Query controller status}
This command array entry is almost identical to the previous entry.  The
only change is that a different custom conversion function is used.
\begin{verbatim}
    {&DSET_LI, GPIBREAD, IB_Q_LOW, "\035", NULL, 0, 10,
        convertStatusReply, 0, 0, NULL, NULL, "\030"},
\end{verbatim}



\subsection{Write the special conversion functions}
As mentioned above, special conversion functions are need to convert reply
messages from the AB300 into EPICS PV values.
The easiest place to put these functions is just before the
\verb@gpibCmds@ table.
The conversion functions are passed a pointer to the \verb@gpibDpvt@ structure
and three values from the command table entry.
The \verb@gpibDpvt@ structure contains a pointer to the EPICS record.
The custom conversion function uses this pointer
to set the record's value field.

Here are the custom conversion functions I wrote for the AB300.
\begin{verbatim}
/*
 * Custom conversion routines
 */
static int
convertPositionReply(struct gpibDpvt *pdpvt, int P1, int P2, char **P3)
{
    struct longinRecord *pli = ((struct longinRecord *)(pdpvt->precord));

    if (pdpvt->msgInputLen != 3) {
        epicsSnprintf(pdpvt->pasynUser->errorMessage,
                      pdpvt->pasynUser->errorMessageSize,
                      "Invalid reply");
        return -1;
    }
    pli->val = pdpvt->msg[0];
    return 0;
}
static int
convertStatusReply(struct gpibDpvt *pdpvt, int P1, int P2, char **P3)
{
    struct longinRecord *pli = ((struct longinRecord *)(pdpvt->precord));

    if (pdpvt->msgInputLen != 3) {
        epicsSnprintf(pdpvt->pasynUser->errorMessage,
                      pdpvt->pasynUser->errorMessageSize,
                      "Invalid reply");
        return -1;
    }
    pli->val = pdpvt->msg[1];
    return 0;
}
\end{verbatim}

Some points of interest:
\begin{enumerate}
\item
The custom conversion function can pass back an error message by printing
it to the \verb@errorMessage@ buffer.
\item
I put in a sanity check to ensure that the end-of-string character is where
it should be.
\end{enumerate}





\subsection{Provide the device support initialization}
Because of way code is stored in object libraries on different systems
the device support parameter table must be initialized at run-time.  The
analog-in initializer is used to perform this operation.  This is why
all device support files must declare an analog-in DSET.

Here's the initialization for the AB300 device support.  As you can see, most of
the skeleton file values are left unchanged:
\begin{alltt}
static long init_ai(int parm)
\verb@{@
  if(parm==0) \verb@{@
    devSupParms.name = "devAB300";
    devSupParms.gpibCmds = gpibCmds;
    devSupParms.numparams = NUMPARAMS;
    devSupParms.timeout = TIMEOUT;
    devSupParms.timeWindow = TIMEWINDOW;
    devSupParms.respond2Writes = 0;
  \verb@}@
  return(0);
\verb@}@
\end{alltt}
Three values have been changed:
\begin{enumerate}
\item
The AB300 sends back values in response to commands, but needs no time delay,
so the respond2Writes entry is set to 0.
\item
The name entry is used for diagnostic purposes only.
\end{enumerate}

\section{Add the device support to the application}
The makeBaseApp.pl script produces an application Makefile
({\tt AB300App/src/Makefile})
with a commented-out set of application example source files.
Remove the comment character and change the example names to the
name of the device support file created in the previous section:
\begin{verbatim}
AB300_SRCS += devAB300.c
\end{verbatim}

You must also link the GPIB support libraries with your application.  Add the following line
\begin{verbatim}
AB300_LIBS += asyn
\end{verbatim}
before the
\begin{verbatim}
AB300_LIBS += $(EPICS_BASE_IOC_LIBS)
\end{verbatim}
line in the application Makefile.





\section{Modify the application database definition file}
Here's where you specify the link between the DSET names defined in the
device support file and the DTYP fields in the application database.
The {\tt AB300App/src/AB300Include.dbd} file created by makeBaseApp.pl needs to
be changed to include this information.
I used \verb@"AB300Gpib"@ as the device type.

The driver support for serial line 'GPIB' devices must also be included in
the application as shown.

\verbatimfile{AB300/AB300App/src/AB300Include.dbd}




\section{Create the application database file}
Now that the application includes the necessary device and driver support 
it's possible to create the database describing the actual EPICS process
variables associated with the filter wheel.  

I created the file {\tt AB300App/Db/AB300.db} with the following contents:
\verbatimfile{AB300/AB300App/Db/AB300.db}
Notes:
\begin{enumerate}
\item The numbers following the \verb@L@ in
the INP and OUT fields are the 
number of the `link' used to communicate with the filter wheel.  This
link is set up at run time by commands in the application startup script.
\item The numbers following the \verb@A@ in
the INP and OUT fields are unused by the device support but must be
a valid GPIB address (0-30) since the GPIB address conversion routines
check the value and the diagnostic display routines require a matching value.
\item The numbers following the \verb|@| in
the INP and OUT fields are the 
indices into the GPIB command array.
\item The DTYP fields must match the names specified in the AB300Include.dbd
database definition.
\end{enumerate}


\section{Add the database file to the application}
The makeBaseApp.pl script put and exampl application database file
into {\tt AB300App/Db/Makefile} as a comment.  
Replace the example name with the name of the database file created in the
previous step, leaving:
\begin{verbatim}
DB += AB300.db
\end{verbatim}



\section{Modify the application startup script}
The {\tt iocBoot/iocAB300/st.cmd} application startup script created by the
makeBaseApp.pl script needs a few changes to get the application 
working properly.

\begin{enumerate}
\item Ensure that the application database records are loaded.  Remove the
\verb@#@
and give a reasonable value to the 'user' macro:
\begin{verbatim}
dbLoadRecords("../../db/AB300.db","user=AB300")
\end{verbatim}

\item Set up the 'link' between the IOC and the filter wheel.
\begin{itemize}
\item If you're using
an Ethernet/RS-232 converter or a device which communicates over
a telnet-style socket connection you need to specify the Internet host and
port number like:
\begin{verbatim}
drvGenericSerialConfigure("L0","164.54.9.91:4002",0,0)
\end{verbatim}


\item If you're using a serial line directly attached to the IOC you need
something like:
\begin{verbatim}
drvGenericSerialConfigure("L0","/dev/ttyS0",0,0)
asynSetPortOption("L0", "baud", "9600")
asynSetPortOption("L0", "bits", "8")
asynSetPortOption("L0", "parity", "none")
asynSetPortOption("L0", "stop", "1")
asynSetPortOption("L0", "clocal", "Y")
asynSetPortOption("L0", "crtscts", "N")
\end{verbatim}

\item If you're using a serial line directly attached to a vxWorks IOC you must
first configure the serial port interface hardware.  The following example shows the
commands to configure a port on a GreenSprings UART Industry-Pack module.
\begin{verbatim}
ipacAddVIPC616_01("0x6000,B0000000")
tyGSOctalDrv(1)
tyGSOctalModuleInit("RS232", 0x80, 0, 0)
tyGSOctalDevCreate("/tyGS/0/0",0,0,1000,1000)
drvGenericSerialConfigure("L0","/tyGS/0/0",0,0)
asynSetPortOption("L0","baud","9600")
\end{verbatim}
\end{itemize}

In all of the above examples the first argument of the
drvGenericSerialConfigure and asynSetPortOption commands is
the link identifier and must match the \verb@L@ value in the EPICS
database record INP and OUT fields.

\item (Optional) Add lines to control the debugging level of the serial line 'GPIB' driver.
The following turns on full tracing of control flow and every I/O operation
which is useful during initial debugging.
\begin{verbatim}
asynSetTraceMask("L0",0,0xff)
asynSetTraceIOMask("L0",0,0x2)
\end{verbatim}
A better way to control the amount and type of diagnostic output is through
the use of the
\htmladdnormallink{asynTrace}{../../asynDriver.html#DiagnosticAids}
facility.

\end{enumerate}






\section{Build the application}
Change directories to the top-level directory of your application and:
\begin{alltt}
norume> {\bf make}
\end{alltt}
({\bf gmake} on solaris).

If all goes well you'll be left with an executable program in
{\tt bin/linux-x86/AB300}.



\section{Run the application}
Change directories to where makeBaseApp.pl put the application startup script
and run the application:
\begin{alltt}
norume> {\bf cd iocBoot/iocAB300}
norume> {\bf ../../bin/linux-x86/AB300 st.cmd}
dbLoadDatabase("../../dbd/AB300.dbd",0,0)
AB300_registerRecordDeviceDriver(pdbbase) 
dbLoadRecords("../../db/AB300.db","user=AB300")
drvGenericSerialConfigure("L0","164.54.9.91:4002",0,0)
asynSetTraceMask("L0",0,0xff)
asynSetTraceIOMask("L0",0,0x2)
iocInit()
############################################################################
###  EPICS IOC CORE built on Jan 29 2004
###  EPICS R3.14.5 $$Name: not supported by cvs2svn $$ $$Date: 2004-02-05 18:56:09 $$
############################################################################
Starting iocInit
iocInit: All initialization complete
epics>
\end{alltt}

Check the process variable names:
\begin{alltt}
epics> {\bf dbl}
AB300:FilterWheel:fbk
AB300:FilterWheel:status
AB300:FilterWheel
AB300:FilterWheel:reset
\end{alltt}

Reset the filter wheel.  The values sent between the IOC and the filter wheel
are shown:
\begin{alltt}
epics> {\bf dbpf AB300:FilterWheel:reset 0}
DBR_LONG:           0         0x0                 
drvGenericSerial set eos 1 \verb@\@033
drvGenericSerialWrite 3 \verb@\@377\verb@\@377\verb@\@033
drvGenericSerialRead 1 \verb@\@033
\end{alltt}

Read back the filter wheel position.  The dbtr command prints the record
before the I/O has a chance to occur:
\begin{alltt}
epics> {\bf dbtr AB300:FilterWheel:fbk}
ACKS: NO_ALARM      ACKT: YES           ADEL: 0             ALST: 0
ASG:                BKPT: 0x00          DESC: Filter Wheel Position
DISA: 0             DISP: 0             DISS: NO_ALARM      DISV: 1
DTYP: AB300Gpib     EGU:                EVNT: 0             FLNK:CONSTANT 0
HHSV: NO_ALARM      HIGH: 0             HIHI: 0             HOPR: 6
HSV: NO_ALARM       HYST: 0             INP:GPIB_IO #L0 A0 @2
LALM: 0             LCNT: 0             LLSV: NO_ALARM      LOLO: 0
LOPR: 1             LOW: 0              LSV: NO_ALARM       MDEL: 0
MLST: 0             NAME: AB300:FilterWheel:fbk             NSEV: NO_ALARM
NSTA: NO_ALARM      PACT: 1             PHAS: 0             PINI: NO
PRIO: LOW           PROC: 0             PUTF: 0             RPRO: 0
SCAN: Passive       SDIS:CONSTANT       SEVR: INVALID       SIML:CONSTANT
SIMM: NO            SIMS: NO_ALARM      SIOL:CONSTANT       STAT: UDF
SVAL: 0             TPRO: 0             TSE: 0              TSEL:CONSTANT
UDF: 1              VAL: 0
drvGenericSerial set eos 1 \verb@\@030
drvGenericSerialWrite 1 \verb@\@035
drvGenericSerialRead 3 \verb@\@001\verb@\@020\verb@\@030
\end{alltt}

Now the process variable should have that value:
\begin{alltt}
epics> {\bf dbpr AB300:FilterWheel:fbk}
ASG:                DESC: Filter Wheel Position             DISA: 0
DISP: 0             DISV: 1             NAME: AB300:FilterWheel:fbk
SEVR: NO_ALARM      STAT: NO_ALARM      SVAL: 0             TPRO: 0
VAL: 1
\end{alltt}



Move the wheel to position 4:

\begin{alltt}
epics> {\bf dbpf AB300:FilterWheel 4}
DBR_LONG:           4         0x4
drvGenericSerialWrite 2 \verb@\@017\verb@\@004
drvGenericSerialRead 1 \verb@\@020
drvGenericSerialRead 1 \verb@\@030
\end{alltt}

Read back the position:
\begin{alltt}
epics> {\bf dbtr AB300:FilterWheel:fbk}
ACKS: NO_ALARM      ACKT: YES           ADEL: 0             ALST: 1
ASG:                BKPT: 0x00          DESC: Filter Wheel Position
DISA: 0             DISP: 0             DISS: NO_ALARM      DISV: 1
DTYP: AB300Gpib     EGU:                EVNT: 0             FLNK:CONSTANT 0
HHSV: NO_ALARM      HIGH: 0             HIHI: 0             HOPR: 6
HSV: NO_ALARM       HYST: 0             INP:GPIB_IO #L0 A0 @2
LALM: 1             LCNT: 0             LLSV: NO_ALARM      LOLO: 0
LOPR: 1             LOW: 0              LSV: NO_ALARM       MDEL: 0
MLST: 1             NAME: AB300:FilterWheel:fbk             NSEV: NO_ALARM
NSTA: NO_ALARM      PACT: 1             PHAS: 0             PINI: NO
PRIO: LOW           PROC: 0             PUTF: 0             RPRO: 0
SCAN: Passive       SDIS:CONSTANT       SEVR: NO_ALARM      SIML:CONSTANT
SIMM: NO            SIMS: NO_ALARM      SIOL:CONSTANT       STAT: NO_ALARM
SVAL: 0             TPRO: 0             TSE: 0              TSEL:CONSTANT
UDF: 0              VAL: 1
drvGenericSerialWrite 1 \verb@\@035
drvGenericSerialRead 3 \verb@\@004\verb@\@020\verb@\@030
\end{alltt}

And it really is 4:
\begin{alltt}
epics> {\bf dbpr AB300:FilterWheel:fbk}
ASG:                DESC: Filter Wheel Position             DISA: 0
DISP: 0             DISV: 1             NAME: AB300:FilterWheel:fbk
SEVR: NO_ALARM      STAT: NO_ALARM      SVAL: 0             TPRO: 0
VAL: 4
\end{alltt}

\newpage
\section{Device Support File}
Here is the complete device support file for the AB300 filter wheel
({\tt AB300App/src/devAB300.c}):
\verbatimfile{AB300/AB300App/src/devAB300.c}

\section{asynTrace support}
The asynTrace facility provides a convenient mechanism for controlling
the diagnostic messages produced by asyn drivers.  To add asynTrace
to your application:
\begin{enumerate}
\item  Add the line
\begin{verbatim}
include "devAsynTrace.dbd"
\end{verbatim}
to the application {\it xxx}{\tt Include.dbd} file.

\item  Add the line
\begin{verbatim}
DB_INSTALLS += $(ASYN)/db/asynTrace.db
\end{verbatim}
to an application {\tt Makefile}.

\item  For each port on which you wish to control diagnostic messages,
add a line like
\begin{verbatim}
dbLoadRecords("db/asynTrace.db","ioc=ab300Test,port=L0,addr=0")
\end{verbatim}
to the application startup ({\tt st.cmd}) script.
The {\tt port} value must match the the value in the corresponding
{\tt drvGenericSerialConfigure} command. The {\tt addr} value should be zero.
The {\tt ioc} value is arbitrary and should be chosen so that the ioc/port/addr
combination is unique among all IOCs on your network.
\end{enumerate}

To run the asynTrace screen, add {\it $<$asynTop$>$}{\tt /medm} to your
\verb@EPICS_DISPLAY_PATH@ environment variable and start medm with
{\tt ioc}, {\tt port} and {\tt addr} values
matching those given in the {\tt dbLoadRecords} command:
\begin{verbatim}
medm -x -macro "ioc=ab300Test,port=L0,addr=0" asynTrace.adl &
\end{verbatim}


\end{document}
